\documentclass[12pt]{article}

\usepackage[latin1]{inputenc}
\usepackage{amssymb}
\usepackage{amsmath}
\usepackage{amsthm}
\usepackage{latexsym} 
\usepackage{graphicx}
\usepackage{bm}  
\usepackage{overpic} 
\usepackage[normalem]{ulem}
  
\usepackage{exscale}
\usepackage{amsfonts}
\usepackage[usenames,dvipsnames]{color} % load color package

\textwidth=6.0in \textheight=8.8in \hoffset=-0.2in
\voffset=-0.85in
\parskip=6pt
\baselineskip=9pt
\topmargin 0.8in
 
\def\black#1{\textcolor{black}{#1}}
\def\blue#1{\textcolor{blue}{#1}}
\def\red#1{\textcolor{red}{#1}}
\def\green#1{\textcolor{green}{#1}}
\def\yellow#1{\textcolor{yellow}{#1}}
\def\orange{\textcolor{BurntOrange}}

\newtheorem{definition}{Definition}[section]
\newtheorem{lemma}{Lemma}[section]
\newtheorem{remark}{Remark}[section]
\newtheorem{example}{Example}[section]
\newtheorem{theorem}{Theorem}[section]
\newtheorem{cor}{Corollary}[section]
\newtheorem{corollary}{Corollary}[section]

\numberwithin{equation}{section}

\newcommand{\E}{\mathbb{E}}
\newcommand{\R}{\mathbb{R}}
\newcommand{\sigl}{\sigma_L}
\newcommand{\BS}{\rm BS}
\newcommand{\p}{\partial}
\newcommand{\var}{{\rm var}}
\newcommand{\cov}{{\rm cov}}
\newcommand{\beaa}{\begin{eqnarray*}}
\newcommand{\eeaa}{\end{eqnarray*}}
\newcommand{\bea}{\begin{eqnarray}}
\newcommand{\eea}{\end{eqnarray}}
\newcommand{\ben}{\begin{enumerate}}
\newcommand{\een}{\end{enumerate}}


\def\cC{\mathcal C}
\def\cD{\mathcal D}
\def\cS{\mathcal S}
\def\cH{\mathcal H}
\def\cI{\mathcal I}
\def\cJ{\mathcal J}
\def\cL{\mathcal L}
\def\cV{\mathcal V}
\def\cR{\mathcal R}
\def\bR{\mathbb R}
\def\cX{\mathcal X}
\def\cF{\mathcal F}
\def\bP{\mathbb P}
\def\bE{\mathbb E}
\def\bN{\mathbb N}
\def\bT{\mathbb T}
\def\bC{\mathbb C}
\def\var{\text{var\,}}
\def\eps{\varepsilon}

\newcommand{\mt}{\mathbf{t}}
\newcommand{\mS}{\mathbf{S}}
\newcommand{\tC}{\widetilde{C}}
\newcommand{\hC}{\widehat{C}}
\newcommand{\tH}{\widetilde{H}}
\renewcommand{\O}{\mathcal{O}}
\newcommand{\dt}{\Delta t}
\newcommand{\tr}{{\rm tr}}

\begin{document}



\title{\bf Capstone report}

\author{Tianrong Wang\footnote{Department of Mathematics, Baruch College, CUNY. {\tt  Tianrong.Wang@baruch.cuny.edu}}{\setcounter{footnote}{1}} , Thomas George\footnote{Department of Mathematics, Baruch College, CUNY. {\tt  Thomas.George@baruch.cuny.edu}}{\setcounter{footnote}{2}} \thanks{We wish to thank Professor Gatheral for his guidance and supervision . We also wish to thank Mr. Reghai for his constant support}
}

%\date{This version: December 25, 2011}


\maketitle\thispagestyle{empty}
 
%%***************************************************************************
%%
%%  Document begins here
%%
%%***************************************************************************



\begin{abstract}
In this report, we describe our final project ....
\end{abstract}

%%%%%%%%%%%%%%%%%%%%%%%%%%%%%%%%%%%%%%%%%%%%%%%%%%%%%%%%%%%%%%%%%%%%%%%%%%%%%%%%%
%
%
%  Section: Introduction
%
%
%%%%%%%%%%%%%%%%%%%%%%%%%%%%%%%%%%%%%%%%%%%%%%%%%%%%%%%%%%%%%%%%%%%%%%%%%%%%%%%%%%

\section{Current Progress}
\subsection{Options pricing models}
\subsubsection{Local Vol surface}
The Local Vol model we are using is the parameterized SVI model provided by professor Gatheral.
\[
    \sigma^2(k,t)=a+b\{\rho (\frac{k}{\sqrt{t}}-m)+\sqrt{(\frac{k}{\sqrt{t}}-m)^2+\sigma^2 t}\}
\]
The parameters are:
\begin{eqnarray*} 
    a&=&0.0012\\
    b&=&0.1634\\
    \sigma&=&0.1029\\
    \rho&=&-0.5555\\
    m&=&0.0439
\end{eqnarray*} 

\begin{figure}[h]
    \includegraphics[width=150mm]{local vol.png}
    \label{fig:LVsurface}
\end{figure}
\subsubsection{Local Stochastic Vol model}
The local stochatic volatility model we are using is
\begin{equation} \label{eq:1}
    \frac{dS_t}{S_t}=\frac{\sigma_D(t,S_t)}{\sqrt{\E(e^{2Y_t^\epsilon}|S_t)}}e^{Y_t^\epsilon}dW_t
\end{equation} 
\[
    dY_t^\epsilon=-\frac \kappa \epsilon Y_t^\epsilon dt + \frac{\mu}{\sqrt{\epsilon}}dB_t
\]
When $\lim_{\epsilon \to 0}$, $Y_t^\epsilon\sim N(0,\sigma_y^2)$ Where $\sigma_y^2=\frac{\nu^2}{2\kappa}$

Therefore, When $\lim_{\epsilon \to 0}$ \ref{eq:1} can be written as
\begin{equation} \label{eq:2}
    \frac{dS_t}{S_t}=\frac{\sigma_D(t,S_t)}{e^{\sigma_y^2}}e^{Y_t^\epsilon}dW_t
\end{equation} 

\subsubsection{Local Vol model}
The local vol pricing model we are using is based on the process from \ref{eq:2}

\begin{equation} \label{eq:loc}
    \frac{dS_t}{S_t}=\sigma_D(t,S_t)e^{Y-\sigma_y^2}dW_t
\end{equation}

We denote the option price from this process with deformation term $Y=y$ 
as $\pi_{LV}^y$. When there's no deformation $Y=\sigma_y^2$ we denote the option price
as $\pi_{LV}$.

\subsection{Numerical result}
The local vol model we are using in this part is \ref{eq:loc}

The theoretical LSV impact is 
\begin{equation} \label{eq:3}
    \pi_{LSV}-\pi_{LV}=\frac 1 2 \sigma^2_y \frac{\partial_E^2 \pi^\beta_{LV}}{\partial \beta^2}|_{\beta=\sigma^2_y}
    +\frac{\rho \nu}{\kappa}\frac{\partial_E^2 \pi_{LV}}{\partial \ln S_0\partial \sigma}
\end{equation} 
Where
\begin{equation} \label{eq:4}
    \frac{\partial_E^2 \pi^\beta_{LV}}{\partial \beta^2}=\frac{\partial^2 \pi^\beta_{LV}}{\partial \beta^2}
    -\int_0^T\int_0^\infty (q_{K,T}\frac{\partial^2 C^{\beta,K,T}_{LV}}{\partial \beta^2})dKdT\\
\end{equation} 

\begin{equation} \label{eq:5}
    \frac{\partial_E^2 \pi_{LV}}{\partial \ln S_0\partial \sigma}=\frac{\partial^2\pi_{LV}}{\partial \ln S_0\partial \sigma}
    -\int_0^T\int_0^\infty (q_{K,T}\frac{\partial^2 C^{K,T}_{LV}}{\partial \ln S_0\partial \sigma})dKdT
\end{equation} 
\subsubsection{Special scenarii to simplify computation}
We remove the double integral in the above section by designing special scenarii

For exotic Volga, for the vanilla effect, we try to use an asymetric bump $x_{K,T}$ for different strikes and maturities to remove it
\begin{equation} \label{eq:8}
    C_{LV}^{K,T,\sigma_y^2+bump}-2 C_{LV}^{K,T,\sigma_y^2}+C_{LV}^{K,T,\sigma_y^2-beta}=0
\end{equation} 
The effect of adding this asymetric bump can be modeled as moving the local volatility surface, because this bump is 
a multiplier for each point on the local vol surface $\sigma_D(t,S_t)$ as we showed in \ref{eq:loc}\\
We used a Black-Scholes solver to compute this new surface by using the following formula on each strike and maturity
\begin{equation} 
    \sigma_{BS,K,T}+x_{K,T}=C^{-1}(2 C_{LV}^{K,T,\sigma_y^2}-C_{LV}^{K,T,\sigma_y^2-beta})=0
\end{equation} 
\begin{figure}[h]
    \includegraphics[width=150mm]{volga x_kt.png}
\end{figure}
$x_{K,T}$ is used as the bump for each $K,T$ on the local vol surface.
This method used the approximation that the bump on the local vol surface is the same 
as the resulted bump on the implied vol surface.\\

After getting the bumps for each strike and maturity, we construct a new local vol surface $\sigma_{K,T}+x_{K,T}$.
The exotic volga is computed in the following way:

\begin{equation} \label{eq:6}
    \frac{\partial_E^2 \pi^\beta_{LV}}{\partial \beta^2}=\lim_{\beta \to 0}\frac{\pi_{LV}(\sigma_{K,T}+x_{K,T})-2\pi_{LV}+\pi_{LV}^{-\beta}}{\beta^2}\\
\end{equation} 


The exotic vanna is computed in a similar way. We first get the bumps $x_{K,T}$ for each strike and maturity by using the following equation
\begin{equation} \label{eq:9}
    0=C_{LV}^{K,T}(S+\delta S,\sigma_{K,T}+\delta \sigma)-C_{LV}^{K,T}(S+\delta S,\sigma_{K,T})-C_{LV}^{K,T}(S,\sigma_{K,T}+x_{K,T})+C_{LV}^{K,T}(S,\sigma_{K,T})
\end{equation} 
$C_{LV}^{K,T}(S+\delta S,\sigma_{K,T}+\delta \sigma)$ means a small change on spot price and an
uniform bump on the whole local vol surface.
Then we solve $x_{K,T}$, which is a single point bump by using Black-Scholes solver on each K and T

\[
    \sigma_{BS,K,T}+x_{K,T}=C_{LV}^{K,T}(S+\delta S,\sigma_{K,T}+\delta \sigma)-C_{LV}^{K,T}(S+\delta S,\sigma_{K,T})+C_{LV}^{K,T}(S,\sigma_{K,T})
\]
\begin{figure}[h]
    \includegraphics[width=150mm]{vanna x_kt.png}
\end{figure}
We again used the bump on the implied vol surface to apprximate the bump on the local vol surface.

Exotic vanna is computed in the following way
\begin{eqnarray*} 
    &&\frac{\partial_E^2 \pi_{LV}}{\partial \ln S_0 \partial \sigma}\\
    &=&S_0\frac{\pi_{LV}(S_0+\delta S ,\sigma_{K,T}+\delta \sigma)-\pi_{LV}(S_0+\delta S ,\sigma_{K,T})-\pi_{LV}(S_0,\sigma_{K,T}+x_{K,T})+\pi_{LV}(S_0,\sigma_{K,T})}
    {\delta S \delta \sigma}
\end{eqnarray*} 


\subsection{Current result}
In the experiment we use down-and-out call options with barrier $B=0.9$, spot price $S_0=1$ , log strikes $k=log(K/S_0) \in (-0.6,0.2)$. LSV parameters $\rho=-0.9$, $nu=0.1$, $kappa=10$. The LSV impact we get is as figure \ref{fig:LSV impact}. The actual LSV-LV difference is as figure \ref{fig:LSV-LV}
When getting Greeks, we used $\beta=0.03$, $\delta S=0.03$, $\delta \sigma=0.03$.
\begin{figure}[h]
    \includegraphics[width=150mm]{LSV impact.png}
    \label{fig:LSV impact}
\end{figure}

\begin{figure}[h]
    \includegraphics[width=150mm]{LSV minus LV.png}
    \label{fig:LSV-LV}
\end{figure}

\section{Math proofs for the formulas}

\subsection{Singular pertubation}
Considering the LSV process \ref{eq:1} as $lim{\epsilon to 0}$. 
Suppose the option price is $u(t,x,y)$. The payoff at maturity is $ u(T,x,y)=h(T,x)$.
By applying Feyman-Kac, we obtain the following PDE
\begin{equation}\label{eq:PDE}
    u_t+\frac 1 2 \sigma_D^2(t,x)x^2e^{2(y-\sigma_y^2)}u_xx+\frac{1}{\epsilon}\mathcal{L}_y u
    +\frac{1}{\sqrt{\epsilon}}\rho \nu x \sigma_D^2(t,x)e^{y-\sigma_y^2} u_{xy}
\end{equation}

Where $\mathcal{L}_y=-\kappa \partial_y +\frac 1 2 \nu^2 \partial_{yy}$

We can make an expansion of $u$ as $u=u_0+\sqrt{\epsilon}u_1+\epsilon u_2$, where $u_0$ is the LV price $\pi_{LV}$

By substituting this expansion into the PDE \ref{eq:PDE}, we can match power of $\epsilon$
and solve the system of PDEs.

\begin{eqnarray*} 
&O(\frac{1}{\epsilon}): &\mathcal{L}_y u_0=0  , u_0(T,x,y)=h(T,x)\\
&O(\frac{1}{\sqrt\epsilon}): &\mathcal{L}_y u_1 +\rho \nu x \sigma_D(t,x)e^{y-\sigma_y^2}(u_0)_{xy}=0 , u_1(T,x,y)=0\\
&O(1): &\mathcal{L}_y u_2 +\rho \nu x \sigma_D(t,x)e^{y-\sigma_y^2}(u_1)_{xy}+(\partial_t+\frac 1 2 x^2 \sigma^2_D(t,x)e^{2(y-\sigma_y^2)})u_0=0 , u_2(T,x,y)=0\\
&O(\sqrt\epsilon): &\mathcal{L}_y u_3 +\rho \nu x \sigma_D(t,x)e^{y-\sigma_y^2}(u_2)_{xy}+(\partial_t+\frac 1 2 x^2 \sigma^2_D(t,x)e^{2(y-\sigma_y^2)})u_1=0 \\
\end{eqnarray*} 

\subsubsection{Order 1 term}
By solving the PDE system we can get 
\begin{eqnarray*} 
    u_1&=&<e^{y-\sigma_y}\phi'(y)>(-\frac 1 2 \frac{\rho \nu}{\kappa})\int_0^T S_t \sigma_D(t,S_t)\partial_x(S_t^2 \sigma_D^2(t,S_t) \partial_{xx} u_0)dt\\
    &=&<e^{y-\sigma_y}\phi'(y)>(-\frac 1 2 \frac{\rho \nu}{\kappa}) \partial^2_{\ln x, \sigma} u_0
\end{eqnarray*} 

\subsubsection{Order 0 and order 2 term}

The non-calibrated LV process is \ref{eq:loc}

With equation $E(f(Y))\approx f(E(Y))+\frac 1 2 Var(Y) f''(E(Y))$
If we take  $E(f(Y))$ as the non calibrated LV price,then $f(E(Y))$ means the LV price with no deformation.
\begin{equation}
    u_0+u_1\approx\pi_{LV}+\frac 1 2 \sigma^2_y \frac{\partial_E^2 \pi^\beta_{LV}}{\partial \beta^2}|_{\beta=\sigma^2_y}
\end{equation}

\subsection{Modified Newton}
In the above section we get the non-calibrated price:
\begin{equation}
    P^{NC}=\pi_{LV}+\frac 1 2 \sigma^2_y \frac{\partial^2 \pi^\beta_{LV}}{\partial \beta^2}|_{\beta=\sigma^2_y}+\frac{\rho \nu}{\kappa}\frac{\partial^2\pi_{LV}}{\partial \ln S_0\partial \sigma}
\end{equation}
We apply newton's method next to make sure our model fits the vanilla prices
\begin{equation}
    f(x^*)\approx f(x)-\frac{\partial_x f(x)}{\partial_x g(x)}g(x)
\end{equation}
$g(x)=C_{LV}-C_{market}$ is the constrain. 
$\frac{\partial_x f(x)}{\partial_x g(x)}=q_{KT}$ is the change rate of exotic over vanilla as the local vol surface moves
Since we want to match vanilla price for different K and T we got the integral form, numerically it equals to an average.
\begin{equation}
    P=P^{NC}-\int_0^T\int_0^{\infty}(C_{LV}^{K,T}+\frac 1 2 \sigma^2_y \frac{\partial^2 C^{\beta,K,T}_{LV}}{\partial \beta^2}|_{\beta=\sigma^2_y}+\frac{\rho \nu}{\kappa}\frac{\partial^2 C_{LV}^{K,T}}{\partial \ln S_0\partial \sigma}-C_{market}^{K,T})dKdT
\end{equation}
By using $C_{LV}^{K,T}=C_{market}^{K,T}$, we get the result
\begin{equation}
    P-P^{NC}=\frac 1 2 \sigma^2_y \frac{\partial_E^2 \pi^\beta_{LV}}{\partial \beta^2}|_{\beta=\sigma^2_y}
    +\frac{\rho \nu}{\kappa}\frac{\partial_E^2 \pi_{LV}}{\partial \ln S_0\partial \sigma}
\end{equation}
Where the exotic Greeks are defined in \ref{eq:4} \ref{eq:5} 
% \section{Our main result}

% \begin{theorem}\label{thm:GreatTheorem}
% For any given positive integer $n$, there exists at least one integer greater than $n$.
% \end{theorem}

% \begin{proof}
% Consider $m=n+1$.     
% \end{proof}

% \begin{remark} 
% Note just how brilliant Theorem \ref{thm:GreatTheorem} is!
% \end{remark}

% We obtain
% \begin{cor}
% There exists an integer greater than 3.
% \end{cor}

% \section{Another result}


% \section{Numerical experiment}

% \begin{figure}[htb!]
% \begin{center}
% %\includegraphics{SVIarb}
% \caption{This is a graph of something}
% \label{fig:someGraph}
% \end{center}
% \end{figure}



% \section{Summary and conclusion}


%\appendix





% \section*{Acknowledgments}

% We are very grateful to Jane Brown and Janice Smith.

%%%%%%%%%%%%%%%%%%%%%%%%%%%%%%%%%%%%%%%%%%%%%%%%%%%%%%%%%%%%%%%%%%%%%%%%%%%%%%%%%%%%%%%%
%
%
%  Bibliography
%
%
%%%%%%%%%%%%%%%%%%%%%%%%%%%%%%%%%%%%%%%%%%%%%%%%%%%%%%%%%%%%%%%%%%%%%%%%%%%%%%%%%%%%%%%%

% \begin{thebibliography}{}


% % \bibitem{jimbook} { Gatheral, J.},
% % {The Volatility Surface: A Practitioner's Guide},
% % {Wiley Finance} (2006).

% % \bibitem{ghlow}
% % { Gatheral, J.}, { Hsu, E.P.}, { Laurence, P.}, { Ouyang, C.}, and { Wang, T.-H.},
% % {Asymptotics of implied volatility in local volatility models},
% % {\it Mathematical Finance} (2011) forthcoming.



% \end{thebibliography}

\end{document}


