\documentclass[12pt]{article}

\usepackage[latin1]{inputenc}
\usepackage{amssymb}
\usepackage{amsmath}
\usepackage{amsthm}
\usepackage{latexsym} 
\usepackage{graphicx}
\usepackage{bm}  
\usepackage{overpic} 
\usepackage[normalem]{ulem}
  
\usepackage{exscale}
\usepackage{amsfonts}
\usepackage[usenames,dvipsnames]{color} % load color package

\textwidth=6.0in \textheight=8.8in \hoffset=-0.2in
\voffset=-0.85in
\parskip=6pt
\baselineskip=9pt
\topmargin 0.8in
 
\def\black#1{\textcolor{black}{#1}}
\def\blue#1{\textcolor{blue}{#1}}
\def\red#1{\textcolor{red}{#1}}
\def\green#1{\textcolor{green}{#1}}
\def\yellow#1{\textcolor{yellow}{#1}}
\def\orange{\textcolor{BurntOrange}}

\newtheorem{definition}{Definition}[section]
\newtheorem{lemma}{Lemma}[section]
\newtheorem{remark}{Remark}[section]
\newtheorem{example}{Example}[section]
\newtheorem{theorem}{Theorem}[section]
\newtheorem{cor}{Corollary}[section]
\newtheorem{corollary}{Corollary}[section]

\numberwithin{equation}{section}

\newcommand{\E}{\mathbb{E}}
\newcommand{\R}{\mathbb{R}}
\newcommand{\sigl}{\sigma_L}
\newcommand{\BS}{\rm BS}
\newcommand{\p}{\partial}
\newcommand{\var}{{\rm var}}
\newcommand{\cov}{{\rm cov}}
\newcommand{\beaa}{\begin{eqnarray*}}
\newcommand{\eeaa}{\end{eqnarray*}}
\newcommand{\bea}{\begin{eqnarray}}
\newcommand{\eea}{\end{eqnarray}}
\newcommand{\ben}{\begin{enumerate}}
\newcommand{\een}{\end{enumerate}}


\def\cC{\mathcal C}
\def\cD{\mathcal D}
\def\cS{\mathcal S}
\def\cH{\mathcal H}
\def\cI{\mathcal I}
\def\cJ{\mathcal J}
\def\cL{\mathcal L}
\def\cV{\mathcal V}
\def\cR{\mathcal R}
\def\bR{\mathbb R}
\def\cX{\mathcal X}
\def\cF{\mathcal F}
\def\bP{\mathbb P}
\def\bE{\mathbb E}
\def\bN{\mathbb N}
\def\bT{\mathbb T}
\def\bC{\mathbb C}
\def\var{\text{var\,}}
\def\eps{\varepsilon}

\newcommand{\mt}{\mathbf{t}}
\newcommand{\mS}{\mathbf{S}}
\newcommand{\tC}{\widetilde{C}}
\newcommand{\hC}{\widehat{C}}
\newcommand{\tH}{\widetilde{H}}
\renewcommand{\O}{\mathcal{O}}
\newcommand{\dt}{\Delta t}
\newcommand{\tr}{{\rm tr}}

\begin{document}



\title{\bf Our brilliant masters project final report}

\author{John Brown\footnote{Department of Mathematics, Baruch College, CUNY. {\tt  john.brown@baruch.cuny.edu}}{\setcounter{footnote}{1}} , James Smith\footnote{Department of Mathematics, Baruch College, CUNY. {\tt  James.Smith@baruch.cuny.edu}}{\setcounter{footnote}{2}} \thanks{We wish to thank our hamsters Buster and Butch for their constant affection.}
}

%\date{This version: December 25, 2011}


\maketitle\thispagestyle{empty}
 
%%***************************************************************************
%%
%%  Document begins here
%%
%%***************************************************************************



\begin{abstract}
In this report, we describe our final project ....
\end{abstract}

%%%%%%%%%%%%%%%%%%%%%%%%%%%%%%%%%%%%%%%%%%%%%%%%%%%%%%%%%%%%%%%%%%%%%%%%%%%%%%%%%
%
%
%  Section: Introduction
%
%
%%%%%%%%%%%%%%%%%%%%%%%%%%%%%%%%%%%%%%%%%%%%%%%%%%%%%%%%%%%%%%%%%%%%%%%%%%%%%%%%%%

\section{Current Progress}
\subsection{Local Vol models}
The Local Vol model I am using is the parameterized SVI model provided by professor Gatheral.
$\sigma_{SVI}(k,t)$ where $k$ is the log strike and $t$ is the current time. I modified it to take $S_t$ and $t$ to be used in the Monte-Carlo local vol pricer.
$\sigma_{SVI}(k=log(\frac{S_t}{S_0}),t)$

\subsection{Exotic Volga}
Exotic Volga is computed in this way
\[
    P(\sigma_{KT}-\delta \sigma)-2P(\sigma_{KT})+P(\sigma_{KT}+x_{KT}^{Volga})
\]
$P(\sigma_{KT}-\delta \sigma)$ means the price of an exotic option using a local vol surface with a constant shift.\\
$P(\sigma_{KT})$ means the price of an exotic under a local vol surface.\\
$P(\sigma_{KT}+x_{KT}^{Volga})$ means the price of an exotic under a local vol surface with each $(S_t=K,t=T)$ has a different shift $x_{KT}^{Volga}$\\

$x_{KT}^{Volga}$ is obtained by solving the following equation for each strike price K and each time to maturity T.\\
\[
    0=C(\sigma_{KT}-\delta \sigma)-2C(\sigma_{KT})+C(\sigma_{KT}+x_{KT}^{Volga})
\]
$\implies$
\[
    \sigma_{BS}+x_{KT}^{Volga} \approx C^{-1}(-C(\sigma_{KT}-\delta \sigma)+2C(\sigma_{KT}))
\]
where $\sigma_{BS}$ is the Black-Scholes implied vol when strike is K and time to maturity is T for the vanilla call options , using the local vol surface we have.\\
$C^{-1}$ is the Black-Scholes implied vol solver.\\
$C(\sigma_{KT}-\delta \sigma)$ is the price of a vanilla call with strike K and time to maturity T, under the local vol surface we have with a constand drift $-\delta \sigma$.\\
$C(\sigma_{KT})$ is the price of a vanilla call with strike K and time to maturity T, under the local vol surface we have.\\

\subsection{Current result}
With $S_0=1$ , log strikes $k=log(K/S_0) \in (-0.6,0.2)$ , time to maturity $T \in (0,1)$. The average exotic greeks are concluded in the table \ref{table:exotic greeks}
\begin{table}[]
    \begin{tabular}{lllll}
    \cline{1-3}
    \multicolumn{1}{|l|}{Average Exotic Greeks across K,T} & \multicolumn{1}{l|}{Exotic Volga} & \multicolumn{1}{l|}{Exotic Vanna} &  &  \\ \cline{1-3}
    \multicolumn{1}{|l|}{Vanilla Call}                     & \multicolumn{1}{l|}{-1.8113e-05}  & \multicolumn{1}{l|}{-0.002063}    &  &  \\ \cline{1-3}
    \multicolumn{1}{|l|}{Down and Out Call}                & \multicolumn{1}{l|}{0.0107}       & \multicolumn{1}{l|}{0.0590}       &  &  \\ \cline{1-3}
                                                           &                                   &                                   &  & 
    \end{tabular}
    \caption{exotic greeks}
    \label{table:exotic greeks}
\end{table}


% \section{Our main result}

% \begin{theorem}\label{thm:GreatTheorem}
% For any given positive integer $n$, there exists at least one integer greater than $n$.
% \end{theorem}

% \begin{proof}
% Consider $m=n+1$.     
% \end{proof}

% \begin{remark} 
% Note just how brilliant Theorem \ref{thm:GreatTheorem} is!
% \end{remark}

% We obtain
% \begin{cor}
% There exists an integer greater than 3.
% \end{cor}

% \section{Another result}


% \section{Numerical experiment}

% \begin{figure}[htb!]
% \begin{center}
% %\includegraphics{SVIarb}
% \caption{This is a graph of something}
% \label{fig:someGraph}
% \end{center}
% \end{figure}



% \section{Summary and conclusion}


%\appendix





% \section*{Acknowledgments}

% We are very grateful to Jane Brown and Janice Smith.

%%%%%%%%%%%%%%%%%%%%%%%%%%%%%%%%%%%%%%%%%%%%%%%%%%%%%%%%%%%%%%%%%%%%%%%%%%%%%%%%%%%%%%%%
%
%
%  Bibliography
%
%
%%%%%%%%%%%%%%%%%%%%%%%%%%%%%%%%%%%%%%%%%%%%%%%%%%%%%%%%%%%%%%%%%%%%%%%%%%%%%%%%%%%%%%%%

\begin{thebibliography}{}


\bibitem{jimbook} { Gatheral, J.},
{The Volatility Surface: A Practitioner's Guide},
{Wiley Finance} (2006).

\bibitem{ghlow}
{ Gatheral, J.}, { Hsu, E.P.}, { Laurence, P.}, { Ouyang, C.}, and { Wang, T.-H.},
{Asymptotics of implied volatility in local volatility models},
{\it Mathematical Finance} (2011) forthcoming.



\end{thebibliography}

\end{document}


